\documentclass[a4paper,10pt]{scrartcl}
\usepackage[utf8]{inputenc}
\usepackage{url}

%opening
\title{Bits and Bites}
\subtitle{A Cooking Recipe Creating System}
\author{Joshua Hopp}
\subject{Computational Creativity Project Outline}

\begin{document}

\maketitle

\section{Why recipes?}
% ultra-short description of the system
% interesting properties of recipes:
%  - algorithm / software analogy
Basically, a cooking recipe is nothing more than an algorithm
or a computer program. There are some basic types of variables describing the input data (called ingrediences)
which are usually declared at the beginning of the recipe and some constants referring to external
ressources (called kitchen tools). The body of the recipe consists of steps. Each step contains one
or more functions (cooking techniques) which are then applied to the input or intermediate products.

Traditional software is written by creative people and interpreted and executed by computers. This projects
allows computers and humans to switch place: The computer will create recipes which can then be interpreted
and executed by humans. 

\section{Creative System Properties}
In order to be creative, the system's output should be novel and useful. Novelty can be achieved by not
repeating existing recipes. Instead, the system should disassemble existing recipes and use them as a
knowledge source. 

The more interesting part is to ensure useful recipes. First, it must be possible to actually cook 
the recipe: Therefore, the ingrediences have to match the cooking techniques applied to them. This
is similar to type checking or type inference other in programming languages.

Then, ingrediences should match based on taste. This is probably way more subjective and probably,
hard constraints should be avoided here to allow the creation of avantguarde recipes. Still, not
everything should be put together.

\section{Time Management}
% point out, how time can be saved as it's only to credits
The time frame to complete the project is one teaching period and the course is worth 2~CP. 
A thorough design and implementation however may take more time than available. Therefore,
time management measures have to be taken to ensure the success of the project.

The problem can be addressed by simplifying the task. Instead of implementing the full system,
the semantic model could be used to display the recipe as a graph or as an assembler-like language.
This requires only the implementation of stages 0 to 3 (see below). Though I really would like to
implement stages 4 and 5, they probably should be considered optional within this project.

\section{Implementation Details}
\setcounter{subsection}{-1}
% introduce all the modules and mention:
%  - purpose and functioning
%  - libraries
%  - time management
The project will be coded in python2.x. The program will consist of several modules or ``stages'':

\newcommand{\paf}{\paragraph{Purpose and Functioning:}}
\newcommand{\lib}{\paragraph{Libraries and Services:}}
\newcommand{\tm}{\paragraph{Time Management and Optional Functionality:}}
\renewcommand{\thesubsection}{Stage \arabic{subsection}:}

\subsection{Database Layer}
\paf Persistent storage and fast execution, especially for \ref{sem} and \ref{ana}. Due to the nature
of the data, a nosql database was chosen instead of forcing the data into a relational scheme.
\lib mongodb (\url{http://www.mongodb.org/}), python-mongoengine

\subsection{Document Retrieval}
\paf To avoid any licensing issues, wikibooks is the source of choice. Content can be incrementally retrieved using the API.
\lib Wikibooks (\url{http://en.wikibooks.org/wiki/Cookbook:Table_of_Contents}),
Mediawiki API (\url{http://en.wikibooks.org/w/api.php})

\subsection{Document and Language Processing}
\paf The document has to be divided into parts. Ingredient list(s) and recipe body (bodies) have to be 
discovered. Some basic language processing has to be applied to the recipies body, in order to get the steps.
\lib python-nltk (\url{http://www.nltk.org/})
\tm Typically, recipes are written in imperative form which simplifies technique recognizion. Typical tools
and ingrediences can be looked up on wikibooks. The language model will be kept as simple as possible. 
Only move to a more complicated model, if the coverage is insufficient.

\subsection{Semantic Model}\label{sem}
\paf A signature has to be assigned to the verbs within the steps, so that the result is clear to the
algorithm. Some might be classified using the wikibooks technique category. All others will have to be
modeled by hand. Also, model what happens to ingrediences after a function is applied.
Display recipes as recipe assembler and nice graphs.
\tm Begin with most frequent verbs, to get a good coverage quickly. This is one of the more important steps.

\subsection{Analysis}\label{ana}
\paf Use data mining algorithms like the A Priori Algorithm to detect common rules in the data. Cluster or tag
ingrediences, techniques and kitchen tools based on the data. Additional data sources can be used to improve the results.
\lib python-numpy, python-sklearn. 
\tm Due to time restrictions, this stage should be considered optional.

\subsection{Recipe Creation}\label{cre}
\paf Start with some random (fitting) ingrediences and apply ``good'' rules. When a valid end node is 
reached, strip the graph and output a new recipe. Also generate ingredient quantities,
preparation time and difficulty.
\tm Due to time restrictions, this stage should be considered optional.

\end{document}
